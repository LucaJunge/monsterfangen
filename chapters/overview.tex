\section{Overview}
%[This section is dedicated to summarize the game and to answer important initial questions: what are the game objectives? What makes it unique? Who is the targeted audience? What is the platform for the game? What genre will the game pertain? What is the overall gameplay? These are samples of what an overview section need to have. The goal of this section is to have a quick way to look for the main highlights of the game. A new member on a game development team can read this section to catch up the idea of the game, or in an advanced game design stage a designer can use it to verify if the ideas he has are in harmony with the general idea of the game. The high concept document can evolve to become this section].

\subsection{Game Abstract}
%[Summarize the game in a few words].

\subsection{Objectives to be achieved by the game}
%[Describe the benefits to be achieved by making the game. Objectives should guide the design decisions of the game. Any constraint should be linked to objectives to].

\subsection{Core gameplay}
%[Describe the main activity the player will be doing in the game. Focus on writing why will it be fun?].

\subsection{Game Features}
%[This section describes the principal characteristics the game will have].

\subsubsection{Genre}
%[Describe the game genre by defining elements or a common basic rule set that describes the nature of the game].

\subsubsection{Number of players}
%[Establish the number of players the game has. If the game has multiplayer describe the number of players intended to handle and indicate if the multiplayer game is competitive, cooperative or collaborative. Describe any special mode the game has for multiplayer].

\subsubsection{Game theme}
%[Describe the guidelines to the aesthetics of the game. Some examples of game themes can be: post nuclear earth, Greek mythology or medieval].

\subsubsection{Story summary}
%[Write a brief summary of the history of the game].
