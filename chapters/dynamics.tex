\section{Dynamics}
%[This section describes the flow of the game. History, levels, chapters, puzzles, interfaces (hardware and software). This section is directly related with the mechanics section since the dynamics are constructed from the elements in the mechanics].

\subsection{Game World}
%[This section describes the world where the game is played].

\subsubsection{Game theme details}
%[Describe the world environment, its ambientation. Put in details how the game world should look, sound and feel].

\subsubsection{Missions / Levels / Chapters flow}
%[Describe how the player can navigate through the world in the game, if navigation is linear or he can choose where to go, if he can skip levels or if there are restrictions to enter in some areas].

\subsubsection{Missions / Levels / Chapters elements}
%[This section describes the elements that will form the core gameplay].

\subsection{Objectives}
%[Describe the objectives to achieve in the dynamics of the game].

\subsection{Rewards}
%[Rewards to the player for his actions in the game. Like achieving a goal or beating a challenge].

\subsection{Challenges}
%[Challenges put to the players throughout the game. Some examples of challenges are: a fight, a puzzle or a boss fight].

\subsection{Special areas}
%[Describe the areas which not classify as mission, level or chapter. Some examples of special areas are: stores, inns or bonus areas].

\subsection{Game interface}
%[Describe every element of every screen that the player can manipulate. Some screen examples can be: title, options, main, inventory or save].

\subsection{Controls interface}
%[Describe how the player can manipulate every screen in the game].

\subsection{Game balance}
%[Describe the elements that are easy to change and can be used to increase or decrease the challenges difficulty. Examples of elements that can easily balance the challenges are enemy speed, life or number of enemies in a fight].